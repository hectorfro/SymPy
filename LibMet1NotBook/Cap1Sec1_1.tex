\documentclass[11pt]{article}
\usepackage[ansinew]{inputenc}
\usepackage[spanish,es-noshorthands]{babel}
    \usepackage[breakable]{tcolorbox}
    \usepackage{parskip} % Stop auto-indenting (to mimic markdown behaviour)
    

    % Basic figure setup, for now with no caption control since it's done
    % automatically by Pandoc (which extracts ![](path) syntax from Markdown).
    \usepackage{graphicx}
    % Maintain compatibility with old templates. Remove in nbconvert 6.0
    \let\Oldincludegraphics\includegraphics
    % Ensure that by default, figures have no caption (until we provide a
    % proper Figure object with a Caption API and a way to capture that
    % in the conversion process - todo).
    \usepackage{caption}
    \DeclareCaptionFormat{nocaption}{}
    \captionsetup{format=nocaption,aboveskip=0pt,belowskip=0pt}

    \usepackage{float}
    \floatplacement{figure}{H} % forces figures to be placed at the correct location
    \usepackage{xcolor} % Allow colors to be defined
    \usepackage{enumerate} % Needed for markdown enumerations to work
    \usepackage{geometry} % Used to adjust the document margins
    \usepackage{amsmath} % Equations
    \usepackage{amssymb} % Equations
    \usepackage{textcomp} % defines textquotesingle
    % Hack from http://tex.stackexchange.com/a/47451/13684:
    \AtBeginDocument{%
        \def\PYZsq{\textquotesingle}% Upright quotes in Pygmentized code
    }
    \usepackage{upquote} % Upright quotes for verbatim code
    \usepackage{eurosym} % defines \euro

    \usepackage{iftex}
    \ifPDFTeX
        \usepackage[T1]{fontenc}
        \IfFileExists{alphabeta.sty}{
              \usepackage{alphabeta}
          }{
              \usepackage[mathletters]{ucs}
              \usepackage[utf8x]{inputenc}
          }
    \else
        \usepackage{fontspec}
        \usepackage{unicode-math}
    \fi

    \usepackage{fancyvrb} % verbatim replacement that allows latex
    \usepackage{grffile} % extends the file name processing of package graphics
                         % to support a larger range
    \makeatletter % fix for old versions of grffile with XeLaTeX
    \@ifpackagelater{grffile}{2019/11/01}
    {
      % Do nothing on new versions
    }
    {
      \def\Gread@@xetex#1{%
        \IfFileExists{"\Gin@base".bb}%
        {\Gread@eps{\Gin@base.bb}}%
        {\Gread@@xetex@aux#1}%
      }
    }
    \makeatother
    \usepackage[Export]{adjustbox} % Used to constrain images to a maximum size
    \adjustboxset{max size={0.9\linewidth}{0.9\paperheight}}

    % The hyperref package gives us a pdf with properly built
    % internal navigation ('pdf bookmarks' for the table of contents,
    % internal cross-reference links, web links for URLs, etc.)
    \usepackage{hyperref}
    % The default LaTeX title has an obnoxious amount of whitespace. By default,
    % titling removes some of it. It also provides customization options.
    \usepackage{titling}
    \usepackage{longtable} % longtable support required by pandoc >1.10
    \usepackage{booktabs}  % table support for pandoc > 1.12.2
    \usepackage{array}     % table support for pandoc >= 2.11.3
    \usepackage{calc}      % table minipage width calculation for pandoc >= 2.11.1
    \usepackage[inline]{enumitem} % IRkernel/repr support (it uses the enumerate* environment)
    \usepackage[normalem]{ulem} % ulem is needed to support strikethroughs (\sout)
                                % normalem makes italics be italics, not underlines
    \usepackage{soul}      % strikethrough (\st) support for pandoc >= 3.0.0
    \usepackage{mathrsfs}
    

    
    % Colors for the hyperref package
    \definecolor{urlcolor}{rgb}{0,.145,.698}
    \definecolor{linkcolor}{rgb}{.71,0.21,0.01}
    \definecolor{citecolor}{rgb}{.12,.54,.11}

    % ANSI colors
    \definecolor{ansi-black}{HTML}{3E424D}
    \definecolor{ansi-black-intense}{HTML}{282C36}
    \definecolor{ansi-red}{HTML}{E75C58}
    \definecolor{ansi-red-intense}{HTML}{B22B31}
    \definecolor{ansi-green}{HTML}{00A250}
    \definecolor{ansi-green-intense}{HTML}{007427}
    \definecolor{ansi-yellow}{HTML}{DDB62B}
    \definecolor{ansi-yellow-intense}{HTML}{B27D12}
    \definecolor{ansi-blue}{HTML}{208FFB}
    \definecolor{ansi-blue-intense}{HTML}{0065CA}
    \definecolor{ansi-magenta}{HTML}{D160C4}
    \definecolor{ansi-magenta-intense}{HTML}{A03196}
    \definecolor{ansi-cyan}{HTML}{60C6C8}
    \definecolor{ansi-cyan-intense}{HTML}{258F8F}
    \definecolor{ansi-white}{HTML}{C5C1B4}
    \definecolor{ansi-white-intense}{HTML}{A1A6B2}
    \definecolor{ansi-default-inverse-fg}{HTML}{FFFFFF}
    \definecolor{ansi-default-inverse-bg}{HTML}{000000}

    % common color for the border for error outputs.
    \definecolor{outerrorbackground}{HTML}{FFDFDF}

    % commands and environments needed by pandoc snippets
    % extracted from the output of `pandoc -s`
    \providecommand{\tightlist}{%
      \setlength{\itemsep}{0pt}\setlength{\parskip}{0pt}}
    \DefineVerbatimEnvironment{Highlighting}{Verbatim}{commandchars=\\\{\}}
    % Add ',fontsize=\small' for more characters per line
    \newenvironment{Shaded}{}{}
    \newcommand{\KeywordTok}[1]{\textcolor[rgb]{0.00,0.44,0.13}{\textbf{{#1}}}}
    \newcommand{\DataTypeTok}[1]{\textcolor[rgb]{0.56,0.13,0.00}{{#1}}}
    \newcommand{\DecValTok}[1]{\textcolor[rgb]{0.25,0.63,0.44}{{#1}}}
    \newcommand{\BaseNTok}[1]{\textcolor[rgb]{0.25,0.63,0.44}{{#1}}}
    \newcommand{\FloatTok}[1]{\textcolor[rgb]{0.25,0.63,0.44}{{#1}}}
    \newcommand{\CharTok}[1]{\textcolor[rgb]{0.25,0.44,0.63}{{#1}}}
    \newcommand{\StringTok}[1]{\textcolor[rgb]{0.25,0.44,0.63}{{#1}}}
    \newcommand{\CommentTok}[1]{\textcolor[rgb]{0.38,0.63,0.69}{\textit{{#1}}}}
    \newcommand{\OtherTok}[1]{\textcolor[rgb]{0.00,0.44,0.13}{{#1}}}
    \newcommand{\AlertTok}[1]{\textcolor[rgb]{1.00,0.00,0.00}{\textbf{{#1}}}}
    \newcommand{\FunctionTok}[1]{\textcolor[rgb]{0.02,0.16,0.49}{{#1}}}
    \newcommand{\RegionMarkerTok}[1]{{#1}}
    \newcommand{\ErrorTok}[1]{\textcolor[rgb]{1.00,0.00,0.00}{\textbf{{#1}}}}
    \newcommand{\NormalTok}[1]{{#1}}

    % Additional commands for more recent versions of Pandoc
    \newcommand{\ConstantTok}[1]{\textcolor[rgb]{0.53,0.00,0.00}{{#1}}}
    \newcommand{\SpecialCharTok}[1]{\textcolor[rgb]{0.25,0.44,0.63}{{#1}}}
    \newcommand{\VerbatimStringTok}[1]{\textcolor[rgb]{0.25,0.44,0.63}{{#1}}}
    \newcommand{\SpecialStringTok}[1]{\textcolor[rgb]{0.73,0.40,0.53}{{#1}}}
    \newcommand{\ImportTok}[1]{{#1}}
    \newcommand{\DocumentationTok}[1]{\textcolor[rgb]{0.73,0.13,0.13}{\textit{{#1}}}}
    \newcommand{\AnnotationTok}[1]{\textcolor[rgb]{0.38,0.63,0.69}{\textbf{\textit{{#1}}}}}
    \newcommand{\CommentVarTok}[1]{\textcolor[rgb]{0.38,0.63,0.69}{\textbf{\textit{{#1}}}}}
    \newcommand{\VariableTok}[1]{\textcolor[rgb]{0.10,0.09,0.49}{{#1}}}
    \newcommand{\ControlFlowTok}[1]{\textcolor[rgb]{0.00,0.44,0.13}{\textbf{{#1}}}}
    \newcommand{\OperatorTok}[1]{\textcolor[rgb]{0.40,0.40,0.40}{{#1}}}
    \newcommand{\BuiltInTok}[1]{{#1}}
    \newcommand{\ExtensionTok}[1]{{#1}}
    \newcommand{\PreprocessorTok}[1]{\textcolor[rgb]{0.74,0.48,0.00}{{#1}}}
    \newcommand{\AttributeTok}[1]{\textcolor[rgb]{0.49,0.56,0.16}{{#1}}}
    \newcommand{\InformationTok}[1]{\textcolor[rgb]{0.38,0.63,0.69}{\textbf{\textit{{#1}}}}}
    \newcommand{\WarningTok}[1]{\textcolor[rgb]{0.38,0.63,0.69}{\textbf{\textit{{#1}}}}}


    % Define a nice break command that doesn't care if a line doesn't already
    % exist.
    \def\br{\hspace*{\fill} \\* }
    % Math Jax compatibility definitions
    \def\gt{>}
    \def\lt{<}
    \let\Oldtex\TeX
    \let\Oldlatex\LaTeX
    \renewcommand{\TeX}{\textrm{\Oldtex}}
    \renewcommand{\LaTeX}{\textrm{\Oldlatex}}
    % Document parameters
    % Document title
    \title{Cap1Sec1\_1}
    
    
    
    
    
    
    
% Pygments definitions
\makeatletter
\def\PY@reset{\let\PY@it=\relax \let\PY@bf=\relax%
    \let\PY@ul=\relax \let\PY@tc=\relax%
    \let\PY@bc=\relax \let\PY@ff=\relax}
\def\PY@tok#1{\csname PY@tok@#1\endcsname}
\def\PY@toks#1+{\ifx\relax#1\empty\else%
    \PY@tok{#1}\expandafter\PY@toks\fi}
\def\PY@do#1{\PY@bc{\PY@tc{\PY@ul{%
    \PY@it{\PY@bf{\PY@ff{#1}}}}}}}
\def\PY#1#2{\PY@reset\PY@toks#1+\relax+\PY@do{#2}}

\@namedef{PY@tok@w}{\def\PY@tc##1{\textcolor[rgb]{0.73,0.73,0.73}{##1}}}
\@namedef{PY@tok@c}{\let\PY@it=\textit\def\PY@tc##1{\textcolor[rgb]{0.24,0.48,0.48}{##1}}}
\@namedef{PY@tok@cp}{\def\PY@tc##1{\textcolor[rgb]{0.61,0.40,0.00}{##1}}}
\@namedef{PY@tok@k}{\let\PY@bf=\textbf\def\PY@tc##1{\textcolor[rgb]{0.00,0.50,0.00}{##1}}}
\@namedef{PY@tok@kp}{\def\PY@tc##1{\textcolor[rgb]{0.00,0.50,0.00}{##1}}}
\@namedef{PY@tok@kt}{\def\PY@tc##1{\textcolor[rgb]{0.69,0.00,0.25}{##1}}}
\@namedef{PY@tok@o}{\def\PY@tc##1{\textcolor[rgb]{0.40,0.40,0.40}{##1}}}
\@namedef{PY@tok@ow}{\let\PY@bf=\textbf\def\PY@tc##1{\textcolor[rgb]{0.67,0.13,1.00}{##1}}}
\@namedef{PY@tok@nb}{\def\PY@tc##1{\textcolor[rgb]{0.00,0.50,0.00}{##1}}}
\@namedef{PY@tok@nf}{\def\PY@tc##1{\textcolor[rgb]{0.00,0.00,1.00}{##1}}}
\@namedef{PY@tok@nc}{\let\PY@bf=\textbf\def\PY@tc##1{\textcolor[rgb]{0.00,0.00,1.00}{##1}}}
\@namedef{PY@tok@nn}{\let\PY@bf=\textbf\def\PY@tc##1{\textcolor[rgb]{0.00,0.00,1.00}{##1}}}
\@namedef{PY@tok@ne}{\let\PY@bf=\textbf\def\PY@tc##1{\textcolor[rgb]{0.80,0.25,0.22}{##1}}}
\@namedef{PY@tok@nv}{\def\PY@tc##1{\textcolor[rgb]{0.10,0.09,0.49}{##1}}}
\@namedef{PY@tok@no}{\def\PY@tc##1{\textcolor[rgb]{0.53,0.00,0.00}{##1}}}
\@namedef{PY@tok@nl}{\def\PY@tc##1{\textcolor[rgb]{0.46,0.46,0.00}{##1}}}
\@namedef{PY@tok@ni}{\let\PY@bf=\textbf\def\PY@tc##1{\textcolor[rgb]{0.44,0.44,0.44}{##1}}}
\@namedef{PY@tok@na}{\def\PY@tc##1{\textcolor[rgb]{0.41,0.47,0.13}{##1}}}
\@namedef{PY@tok@nt}{\let\PY@bf=\textbf\def\PY@tc##1{\textcolor[rgb]{0.00,0.50,0.00}{##1}}}
\@namedef{PY@tok@nd}{\def\PY@tc##1{\textcolor[rgb]{0.67,0.13,1.00}{##1}}}
\@namedef{PY@tok@s}{\def\PY@tc##1{\textcolor[rgb]{0.73,0.13,0.13}{##1}}}
\@namedef{PY@tok@sd}{\let\PY@it=\textit\def\PY@tc##1{\textcolor[rgb]{0.73,0.13,0.13}{##1}}}
\@namedef{PY@tok@si}{\let\PY@bf=\textbf\def\PY@tc##1{\textcolor[rgb]{0.64,0.35,0.47}{##1}}}
\@namedef{PY@tok@se}{\let\PY@bf=\textbf\def\PY@tc##1{\textcolor[rgb]{0.67,0.36,0.12}{##1}}}
\@namedef{PY@tok@sr}{\def\PY@tc##1{\textcolor[rgb]{0.64,0.35,0.47}{##1}}}
\@namedef{PY@tok@ss}{\def\PY@tc##1{\textcolor[rgb]{0.10,0.09,0.49}{##1}}}
\@namedef{PY@tok@sx}{\def\PY@tc##1{\textcolor[rgb]{0.00,0.50,0.00}{##1}}}
\@namedef{PY@tok@m}{\def\PY@tc##1{\textcolor[rgb]{0.40,0.40,0.40}{##1}}}
\@namedef{PY@tok@gh}{\let\PY@bf=\textbf\def\PY@tc##1{\textcolor[rgb]{0.00,0.00,0.50}{##1}}}
\@namedef{PY@tok@gu}{\let\PY@bf=\textbf\def\PY@tc##1{\textcolor[rgb]{0.50,0.00,0.50}{##1}}}
\@namedef{PY@tok@gd}{\def\PY@tc##1{\textcolor[rgb]{0.63,0.00,0.00}{##1}}}
\@namedef{PY@tok@gi}{\def\PY@tc##1{\textcolor[rgb]{0.00,0.52,0.00}{##1}}}
\@namedef{PY@tok@gr}{\def\PY@tc##1{\textcolor[rgb]{0.89,0.00,0.00}{##1}}}
\@namedef{PY@tok@ge}{\let\PY@it=\textit}
\@namedef{PY@tok@gs}{\let\PY@bf=\textbf}
\@namedef{PY@tok@gp}{\let\PY@bf=\textbf\def\PY@tc##1{\textcolor[rgb]{0.00,0.00,0.50}{##1}}}
\@namedef{PY@tok@go}{\def\PY@tc##1{\textcolor[rgb]{0.44,0.44,0.44}{##1}}}
\@namedef{PY@tok@gt}{\def\PY@tc##1{\textcolor[rgb]{0.00,0.27,0.87}{##1}}}
\@namedef{PY@tok@err}{\def\PY@bc##1{{\setlength{\fboxsep}{\string -\fboxrule}\fcolorbox[rgb]{1.00,0.00,0.00}{1,1,1}{\strut ##1}}}}
\@namedef{PY@tok@kc}{\let\PY@bf=\textbf\def\PY@tc##1{\textcolor[rgb]{0.00,0.50,0.00}{##1}}}
\@namedef{PY@tok@kd}{\let\PY@bf=\textbf\def\PY@tc##1{\textcolor[rgb]{0.00,0.50,0.00}{##1}}}
\@namedef{PY@tok@kn}{\let\PY@bf=\textbf\def\PY@tc##1{\textcolor[rgb]{0.00,0.50,0.00}{##1}}}
\@namedef{PY@tok@kr}{\let\PY@bf=\textbf\def\PY@tc##1{\textcolor[rgb]{0.00,0.50,0.00}{##1}}}
\@namedef{PY@tok@bp}{\def\PY@tc##1{\textcolor[rgb]{0.00,0.50,0.00}{##1}}}
\@namedef{PY@tok@fm}{\def\PY@tc##1{\textcolor[rgb]{0.00,0.00,1.00}{##1}}}
\@namedef{PY@tok@vc}{\def\PY@tc##1{\textcolor[rgb]{0.10,0.09,0.49}{##1}}}
\@namedef{PY@tok@vg}{\def\PY@tc##1{\textcolor[rgb]{0.10,0.09,0.49}{##1}}}
\@namedef{PY@tok@vi}{\def\PY@tc##1{\textcolor[rgb]{0.10,0.09,0.49}{##1}}}
\@namedef{PY@tok@vm}{\def\PY@tc##1{\textcolor[rgb]{0.10,0.09,0.49}{##1}}}
\@namedef{PY@tok@sa}{\def\PY@tc##1{\textcolor[rgb]{0.73,0.13,0.13}{##1}}}
\@namedef{PY@tok@sb}{\def\PY@tc##1{\textcolor[rgb]{0.73,0.13,0.13}{##1}}}
\@namedef{PY@tok@sc}{\def\PY@tc##1{\textcolor[rgb]{0.73,0.13,0.13}{##1}}}
\@namedef{PY@tok@dl}{\def\PY@tc##1{\textcolor[rgb]{0.73,0.13,0.13}{##1}}}
\@namedef{PY@tok@s2}{\def\PY@tc##1{\textcolor[rgb]{0.73,0.13,0.13}{##1}}}
\@namedef{PY@tok@sh}{\def\PY@tc##1{\textcolor[rgb]{0.73,0.13,0.13}{##1}}}
\@namedef{PY@tok@s1}{\def\PY@tc##1{\textcolor[rgb]{0.73,0.13,0.13}{##1}}}
\@namedef{PY@tok@mb}{\def\PY@tc##1{\textcolor[rgb]{0.40,0.40,0.40}{##1}}}
\@namedef{PY@tok@mf}{\def\PY@tc##1{\textcolor[rgb]{0.40,0.40,0.40}{##1}}}
\@namedef{PY@tok@mh}{\def\PY@tc##1{\textcolor[rgb]{0.40,0.40,0.40}{##1}}}
\@namedef{PY@tok@mi}{\def\PY@tc##1{\textcolor[rgb]{0.40,0.40,0.40}{##1}}}
\@namedef{PY@tok@il}{\def\PY@tc##1{\textcolor[rgb]{0.40,0.40,0.40}{##1}}}
\@namedef{PY@tok@mo}{\def\PY@tc##1{\textcolor[rgb]{0.40,0.40,0.40}{##1}}}
\@namedef{PY@tok@ch}{\let\PY@it=\textit\def\PY@tc##1{\textcolor[rgb]{0.24,0.48,0.48}{##1}}}
\@namedef{PY@tok@cm}{\let\PY@it=\textit\def\PY@tc##1{\textcolor[rgb]{0.24,0.48,0.48}{##1}}}
\@namedef{PY@tok@cpf}{\let\PY@it=\textit\def\PY@tc##1{\textcolor[rgb]{0.24,0.48,0.48}{##1}}}
\@namedef{PY@tok@c1}{\let\PY@it=\textit\def\PY@tc##1{\textcolor[rgb]{0.24,0.48,0.48}{##1}}}
\@namedef{PY@tok@cs}{\let\PY@it=\textit\def\PY@tc##1{\textcolor[rgb]{0.24,0.48,0.48}{##1}}}

\def\PYZbs{\char`\\}
\def\PYZus{\char`\_}
\def\PYZob{\char`\{}
\def\PYZcb{\char`\}}
\def\PYZca{\char`\^}
\def\PYZam{\char`\&}
\def\PYZlt{\char`\<}
\def\PYZgt{\char`\>}
\def\PYZsh{\char`\#}
\def\PYZpc{\char`\%}
\def\PYZdl{\char`\$}
\def\PYZhy{\char`\-}
\def\PYZsq{\char`\'}
\def\PYZdq{\char`\"}
\def\PYZti{\char`\~}
% for compatibility with earlier versions
\def\PYZat{@}
\def\PYZlb{[}
\def\PYZrb{]}
\makeatother


    % For linebreaks inside Verbatim environment from package fancyvrb.
    \makeatletter
        \newbox\Wrappedcontinuationbox
        \newbox\Wrappedvisiblespacebox
        \newcommand*\Wrappedvisiblespace {\textcolor{red}{\textvisiblespace}}
        \newcommand*\Wrappedcontinuationsymbol {\textcolor{red}{\llap{\tiny$\m@th\hookrightarrow$}}}
        \newcommand*\Wrappedcontinuationindent {3ex }
        \newcommand*\Wrappedafterbreak {\kern\Wrappedcontinuationindent\copy\Wrappedcontinuationbox}
        % Take advantage of the already applied Pygments mark-up to insert
        % potential linebreaks for TeX processing.
        %        {, <, #, %, $, ' and ": go to next line.
        %        _, }, ^, &, >, - and ~: stay at end of broken line.
        % Use of \textquotesingle for straight quote.
        \newcommand*\Wrappedbreaksatspecials {%
            \def\PYGZus{\discretionary{\char`\_}{\Wrappedafterbreak}{\char`\_}}%
            \def\PYGZob{\discretionary{}{\Wrappedafterbreak\char`\{}{\char`\{}}%
            \def\PYGZcb{\discretionary{\char`\}}{\Wrappedafterbreak}{\char`\}}}%
            \def\PYGZca{\discretionary{\char`\^}{\Wrappedafterbreak}{\char`\^}}%
            \def\PYGZam{\discretionary{\char`\&}{\Wrappedafterbreak}{\char`\&}}%
            \def\PYGZlt{\discretionary{}{\Wrappedafterbreak\char`\<}{\char`\<}}%
            \def\PYGZgt{\discretionary{\char`\>}{\Wrappedafterbreak}{\char`\>}}%
            \def\PYGZsh{\discretionary{}{\Wrappedafterbreak\char`\#}{\char`\#}}%
            \def\PYGZpc{\discretionary{}{\Wrappedafterbreak\char`\%}{\char`\%}}%
            \def\PYGZdl{\discretionary{}{\Wrappedafterbreak\char`\$}{\char`\$}}%
            \def\PYGZhy{\discretionary{\char`\-}{\Wrappedafterbreak}{\char`\-}}%
            \def\PYGZsq{\discretionary{}{\Wrappedafterbreak\textquotesingle}{\textquotesingle}}%
            \def\PYGZdq{\discretionary{}{\Wrappedafterbreak\char`\"}{\char`\"}}%
            \def\PYGZti{\discretionary{\char`\~}{\Wrappedafterbreak}{\char`\~}}%
        }
        % Some characters . , ; ? ! / are not pygmentized.
        % This macro makes them "active" and they will insert potential linebreaks
        \newcommand*\Wrappedbreaksatpunct {%
            \lccode`\~`\.\lowercase{\def~}{\discretionary{\hbox{\char`\.}}{\Wrappedafterbreak}{\hbox{\char`\.}}}%
            \lccode`\~`\,\lowercase{\def~}{\discretionary{\hbox{\char`\,}}{\Wrappedafterbreak}{\hbox{\char`\,}}}%
            \lccode`\~`\;\lowercase{\def~}{\discretionary{\hbox{\char`\;}}{\Wrappedafterbreak}{\hbox{\char`\;}}}%
            \lccode`\~`\:\lowercase{\def~}{\discretionary{\hbox{\char`\:}}{\Wrappedafterbreak}{\hbox{\char`\:}}}%
            \lccode`\~`\?\lowercase{\def~}{\discretionary{\hbox{\char`\?}}{\Wrappedafterbreak}{\hbox{\char`\?}}}%
            \lccode`\~`\!\lowercase{\def~}{\discretionary{\hbox{\char`\!}}{\Wrappedafterbreak}{\hbox{\char`\!}}}%
            \lccode`\~`\/\lowercase{\def~}{\discretionary{\hbox{\char`\/}}{\Wrappedafterbreak}{\hbox{\char`\/}}}%
            \catcode`\.\active
            \catcode`\,\active
            \catcode`\;\active
            \catcode`\:\active
            \catcode`\?\active
            \catcode`\!\active
            \catcode`\/\active
            \lccode`\~`\~
        }
    \makeatother

    \let\OriginalVerbatim=\Verbatim
    \makeatletter
    \renewcommand{\Verbatim}[1][1]{%
        %\parskip\z@skip
        \sbox\Wrappedcontinuationbox {\Wrappedcontinuationsymbol}%
        \sbox\Wrappedvisiblespacebox {\FV@SetupFont\Wrappedvisiblespace}%
        \def\FancyVerbFormatLine ##1{\hsize\linewidth
            \vtop{\raggedright\hyphenpenalty\z@\exhyphenpenalty\z@
                \doublehyphendemerits\z@\finalhyphendemerits\z@
                \strut ##1\strut}%
        }%
        % If the linebreak is at a space, the latter will be displayed as visible
        % space at end of first line, and a continuation symbol starts next line.
        % Stretch/shrink are however usually zero for typewriter font.
        \def\FV@Space {%
            \nobreak\hskip\z@ plus\fontdimen3\font minus\fontdimen4\font
            \discretionary{\copy\Wrappedvisiblespacebox}{\Wrappedafterbreak}
            {\kern\fontdimen2\font}%
        }%

        % Allow breaks at special characters using \PYG... macros.
        \Wrappedbreaksatspecials
        % Breaks at punctuation characters . , ; ? ! and / need catcode=\active
        \OriginalVerbatim[#1,codes*=\Wrappedbreaksatpunct]%
    }
    \makeatother

    % Exact colors from NB
    \definecolor{incolor}{HTML}{303F9F}
    \definecolor{outcolor}{HTML}{D84315}
    \definecolor{cellborder}{HTML}{CFCFCF}
    \definecolor{cellbackground}{HTML}{F7F7F7}

    % prompt
    \makeatletter
    \newcommand{\boxspacing}{\kern\kvtcb@left@rule\kern\kvtcb@boxsep}
    \makeatother
    \newcommand{\prompt}[4]{
        {\ttfamily\llap{{\color{#2}[#3]:\hspace{3pt}#4}}\vspace{-\baselineskip}}
    }
    

    
    % Prevent overflowing lines due to hard-to-break entities
    \sloppy
    % Setup hyperref package
    \hypersetup{
      breaklinks=true,  % so long urls are correctly broken across lines
      colorlinks=true,
      urlcolor=urlcolor,
      linkcolor=linkcolor,
      citecolor=citecolor,
      }
    % Slightly bigger margins than the latex defaults
    
    \geometry{verbose,tmargin=1in,bmargin=1in,lmargin=1in,rmargin=1in}
    
    

\begin{document}
    
%    \maketitle
    
    

    
    \hypertarget{sympy}{%
\section*{SymPy}\label{sympy}}

\textbf{SymPy} es una biblioteca de Python para matem�ticas simb�licas y
esta primera sesi�n tiene como objetivo brindar una introducci�n a SymPy
para usuarios no especializados en el uso de herramientas
computacionales.

La documentaci�n completa est� disponible en: https://www.sympy.org/es/

    Primero que todo debemos incorporar, de la enorme cantidad de librer�as
que existen para Python, la librer�a \textbf{\emph{SymPy}}. Esto se hace
escribiendo lo siguiente:

    \begin{tcolorbox}[breakable, size=fbox, boxrule=1pt, pad at break*=1mm,colback=cellbackground, colframe=cellborder]
\prompt{In}{incolor}{1}{\boxspacing}
\begin{Verbatim}[commandchars=\\\{\}]
\PY{c+c1}{\PYZsh{} Esta linea es un comentario gracias al \PYZsh{}}
\PY{k+kn}{import} \PY{n+nn}{sympy}
\PY{k+kn}{from} \PY{n+nn}{sympy} \PY{k+kn}{import} \PY{o}{*}
\end{Verbatim}
\end{tcolorbox}

    \begin{tcolorbox}[breakable, size=fbox, boxrule=1pt, pad at break*=1mm,colback=cellbackground, colframe=cellborder]
\prompt{In}{incolor}{2}{\boxspacing}
\begin{Verbatim}[commandchars=\\\{\}]
\PY{n}{\PYZus{}\PYZus{}version\PYZus{}\PYZus{}}  \PY{c+c1}{\PYZsh{} Esto es para ver la versi�n de Sympy que estamos usando}
\end{Verbatim}
\end{tcolorbox}

            \begin{tcolorbox}[breakable, size=fbox, boxrule=.5pt, pad at break*=1mm, opacityfill=0]
\prompt{Out}{outcolor}{2}{\boxspacing}
\begin{Verbatim}[commandchars=\\\{\}]
'1.12'
\end{Verbatim}
\end{tcolorbox}
        
    \hypertarget{sintuxe1xis-buxe1sica}{%
\subsection*{Sint�xis b�sica}\label{sintuxe1xis-buxe1sica}}

    Si queremos calcular: \(3!+2^3-\frac12\) debemos escribir:

    \begin{tcolorbox}[breakable, size=fbox, boxrule=1pt, pad at break*=1mm,colback=cellbackground, colframe=cellborder]
\prompt{In}{incolor}{3}{\boxspacing}
\begin{Verbatim}[commandchars=\\\{\}]
\PY{n}{factorial}\PY{p}{(}\PY{l+m+mi}{3}\PY{p}{)} \PY{o}{+} \PY{l+m+mi}{2}\PY{o}{*}\PY{o}{*}\PY{l+m+mi}{3} \PY{o}{\PYZhy{}}  \PY{n}{Rational}\PY{p}{(}\PY{l+m+mi}{1}\PY{p}{,}\PY{l+m+mi}{2}\PY{p}{)}
\end{Verbatim}
\end{tcolorbox}
 
            
\prompt{Out}{outcolor}{3}{}
    
    $\displaystyle \frac{27}{2}$

    

    Note la diferencia en el resultado si escribimos

    \begin{tcolorbox}[breakable, size=fbox, boxrule=1pt, pad at break*=1mm,colback=cellbackground, colframe=cellborder]
\prompt{In}{incolor}{4}{\boxspacing}
\begin{Verbatim}[commandchars=\\\{\}]
\PY{n}{factorial}\PY{p}{(}\PY{l+m+mi}{3}\PY{p}{)} \PY{o}{+} \PY{l+m+mi}{2}\PY{o}{*}\PY{o}{*}\PY{l+m+mi}{3} \PY{o}{\PYZhy{}}  \PY{l+m+mi}{1}\PY{o}{/}\PY{l+m+mi}{2}
\end{Verbatim}
\end{tcolorbox}
 
            
\prompt{Out}{outcolor}{4}{}
    
    $\displaystyle 13.5$

    

    Esto es porque SymPy define tres tipos de números: Real, Racional y
Entero.

    Ahora probemos con el valor de: \(\sqrt{8}\)

    \begin{tcolorbox}[breakable, size=fbox, boxrule=1pt, pad at break*=1mm,colback=cellbackground, colframe=cellborder]
\prompt{In}{incolor}{5}{\boxspacing}
\begin{Verbatim}[commandchars=\\\{\}]
\PY{n}{sqrt}\PY{p}{(}\PY{l+m+mi}{8}\PY{p}{)}
\end{Verbatim}
\end{tcolorbox}
 
            
\prompt{Out}{outcolor}{5}{}
    
    $\displaystyle 2 \sqrt{2}$

    

    Si queremos el valor num�rico podemos usar diferentes funciones de Sympy

    \begin{tcolorbox}[breakable, size=fbox, boxrule=1pt, pad at break*=1mm,colback=cellbackground, colframe=cellborder]
\prompt{In}{incolor}{6}{\boxspacing}
\begin{Verbatim}[commandchars=\\\{\}]
\PY{n+nb}{float}\PY{p}{(}\PY{n}{sqrt}\PY{p}{(}\PY{l+m+mi}{8}\PY{p}{)}\PY{p}{)}
\end{Verbatim}
\end{tcolorbox}

            \begin{tcolorbox}[breakable, size=fbox, boxrule=.5pt, pad at break*=1mm, opacityfill=0]
\prompt{Out}{outcolor}{6}{\boxspacing}
\begin{Verbatim}[commandchars=\\\{\}]
2.8284271247461903
\end{Verbatim}
\end{tcolorbox}
        
    Otras variantes

    \begin{tcolorbox}[breakable, size=fbox, boxrule=1pt, pad at break*=1mm,colback=cellbackground, colframe=cellborder]
\prompt{In}{incolor}{7}{\boxspacing}
\begin{Verbatim}[commandchars=\\\{\}]
\PY{n}{N}\PY{p}{(}\PY{n}{sqrt}\PY{p}{(}\PY{l+m+mi}{8}\PY{p}{)}\PY{p}{,}\PY{l+m+mi}{10}\PY{p}{)}
\end{Verbatim}
\end{tcolorbox}
 
            
\prompt{Out}{outcolor}{7}{}
    
    $\displaystyle 2.828427125$

    

    \begin{tcolorbox}[breakable, size=fbox, boxrule=1pt, pad at break*=1mm,colback=cellbackground, colframe=cellborder]
\prompt{In}{incolor}{8}{\boxspacing}
\begin{Verbatim}[commandchars=\\\{\}]
\PY{n}{sqrt}\PY{p}{(}\PY{l+m+mi}{8}\PY{p}{)}\PY{o}{.}\PY{n}{evalf}\PY{p}{(}\PY{l+m+mi}{10}\PY{p}{)}
\end{Verbatim}
\end{tcolorbox}
 
            
\prompt{Out}{outcolor}{8}{}
    
    $\displaystyle 2.828427125$

    

    \begin{tcolorbox}[breakable, size=fbox, boxrule=1pt, pad at break*=1mm,colback=cellbackground, colframe=cellborder]
\prompt{In}{incolor}{9}{\boxspacing}
\begin{Verbatim}[commandchars=\\\{\}]
\PY{n+nb}{round}\PY{p}{(}\PY{n}{sqrt}\PY{p}{(}\PY{l+m+mi}{8}\PY{p}{)}\PY{p}{,}\PY{l+m+mi}{3}\PY{p}{)}
\end{Verbatim}
\end{tcolorbox}
 
            
\prompt{Out}{outcolor}{9}{}
    
    $\displaystyle 2.828$

    

    \begin{tcolorbox}[breakable, size=fbox, boxrule=1pt, pad at break*=1mm,colback=cellbackground, colframe=cellborder]
\prompt{In}{incolor}{10}{\boxspacing}
\begin{Verbatim}[commandchars=\\\{\}]
\PY{n}{pi} \PY{c+c1}{\PYZsh{} La constante π}
\end{Verbatim}
\end{tcolorbox}
 
            
\prompt{Out}{outcolor}{10}{}
    
    $\displaystyle \pi$

    

    \begin{tcolorbox}[breakable, size=fbox, boxrule=1pt, pad at break*=1mm,colback=cellbackground, colframe=cellborder]
\prompt{In}{incolor}{11}{\boxspacing}
\begin{Verbatim}[commandchars=\\\{\}]
\PY{n}{pi}\PY{o}{.}\PY{n}{evalf}\PY{p}{(}\PY{l+m+mi}{50}\PY{p}{)}
\end{Verbatim}
\end{tcolorbox}
 
            
\prompt{Out}{outcolor}{11}{}
    
    $\displaystyle 3.1415926535897932384626433832795028841971693993751$

    

    El \(\ln(e)\)

    \begin{tcolorbox}[breakable, size=fbox, boxrule=1pt, pad at break*=1mm,colback=cellbackground, colframe=cellborder]
\prompt{In}{incolor}{12}{\boxspacing}
\begin{Verbatim}[commandchars=\\\{\}]
\PY{n}{log}\PY{p}{(}\PY{n}{E}\PY{p}{)}
\end{Verbatim}
\end{tcolorbox}
 
            
\prompt{Out}{outcolor}{12}{}
    
    $\displaystyle 1$

    

    \begin{tcolorbox}[breakable, size=fbox, boxrule=1pt, pad at break*=1mm,colback=cellbackground, colframe=cellborder]
\prompt{In}{incolor}{13}{\boxspacing}
\begin{Verbatim}[commandchars=\\\{\}]
\PY{n}{log}\PY{p}{(}\PY{n}{E}\PY{p}{)}\PY{o}{.}\PY{n}{evalf}\PY{p}{(}\PY{l+m+mi}{2}\PY{p}{)}
\end{Verbatim}
\end{tcolorbox}
 
            
\prompt{Out}{outcolor}{13}{}
    
    $\displaystyle 1.0$

    

    \begin{tcolorbox}[breakable, size=fbox, boxrule=1pt, pad at break*=1mm,colback=cellbackground, colframe=cellborder]
\prompt{In}{incolor}{14}{\boxspacing}
\begin{Verbatim}[commandchars=\\\{\}]
\PY{n}{log}\PY{p}{(}\PY{l+m+mi}{10}\PY{p}{)}
\end{Verbatim}
\end{tcolorbox}
 
            
\prompt{Out}{outcolor}{14}{}
    
    $\displaystyle \log{\left(10 \right)}$

    

    \begin{tcolorbox}[breakable, size=fbox, boxrule=1pt, pad at break*=1mm,colback=cellbackground, colframe=cellborder]
\prompt{In}{incolor}{15}{\boxspacing}
\begin{Verbatim}[commandchars=\\\{\}]
\PY{n}{log}\PY{p}{(}\PY{l+m+mi}{10}\PY{p}{)}\PY{o}{.}\PY{n}{evalf}\PY{p}{(}\PY{l+m+mi}{5}\PY{p}{)}
\end{Verbatim}
\end{tcolorbox}
 
            
\prompt{Out}{outcolor}{15}{}
    
    $\displaystyle 2.3026$

    

    El \(\log_{10}(10)\)

    \begin{tcolorbox}[breakable, size=fbox, boxrule=1pt, pad at break*=1mm,colback=cellbackground, colframe=cellborder]
\prompt{In}{incolor}{16}{\boxspacing}
\begin{Verbatim}[commandchars=\\\{\}]
\PY{n}{log}\PY{p}{(}\PY{l+m+mi}{10}\PY{p}{,}\PY{l+m+mi}{10}\PY{p}{)}
\end{Verbatim}
\end{tcolorbox}
 
            
\prompt{Out}{outcolor}{16}{}
    
    $\displaystyle 1$

    

    Consideremos ahora una combinaci�n de operaciones matem�ticas

    \[
\frac{\sqrt{8}}{3} + \ln(e+1) + \log_{10}(3) + e^{\pi^2} + (e^{\pi})^2 - 6! \sin\left(\frac{\pi}{3}\right)  + \sqrt{-1}
\]

    \begin{tcolorbox}[breakable, size=fbox, boxrule=1pt, pad at break*=1mm,colback=cellbackground, colframe=cellborder]
\prompt{In}{incolor}{17}{\boxspacing}
\begin{Verbatim}[commandchars=\\\{\}]
\PY{n}{sqrt}\PY{p}{(}\PY{l+m+mi}{8}\PY{p}{)}\PY{o}{/}\PY{l+m+mi}{3}\PY{o}{+}\PY{n}{ln}\PY{p}{(}\PY{n}{E}\PY{o}{+}\PY{l+m+mi}{1}\PY{p}{)}\PY{o}{+}\PY{n}{log}\PY{p}{(}\PY{l+m+mi}{3}\PY{p}{,}\PY{l+m+mi}{10}\PY{p}{)}\PY{o}{+}\PY{n}{E}\PY{o}{*}\PY{o}{*}\PY{p}{(}\PY{n}{pi}\PY{o}{*}\PY{o}{*}\PY{l+m+mi}{2}\PY{p}{)}\PY{o}{+}\PY{n}{exp}\PY{p}{(}\PY{n}{pi}\PY{p}{)}\PY{o}{*}\PY{o}{*}\PY{l+m+mi}{2}\PY{o}{\PYZhy{}}\PY{n}{factorial}\PY{p}{(}\PY{l+m+mi}{6}\PY{p}{)}\PY{o}{*}\PY{n}{sin}\PY{p}{(}\PY{n}{pi}\PY{o}{/}\PY{l+m+mi}{3}\PY{p}{)}\PY{o}{+}\PY{n}{sqrt}\PY{p}{(}\PY{o}{\PYZhy{}}\PY{l+m+mi}{1}\PY{p}{)}
\end{Verbatim}
\end{tcolorbox}
 
            
\prompt{Out}{outcolor}{17}{}
    
    $\displaystyle - 360 \sqrt{3} + \frac{\log{\left(3 \right)}}{\log{\left(10 \right)}} + \frac{2 \sqrt{2}}{3} + \log{\left(1 + e \right)} + e^{2 \pi} + e^{\pi^{2}} + i$

    

    Aqu� aprenderemos un atajo. Queremos reutilizar la última salida en el
siguiente comando.

Para hacer esto se utiliza \_ como el argumento del comando. En este
caso el comando que utilizaremos es \textbf{round}, que nos dar� el
valor númerico de la expresi�n anterior con los decimales que
especifiquemos.

    \begin{tcolorbox}[breakable, size=fbox, boxrule=1pt, pad at break*=1mm,colback=cellbackground, colframe=cellborder]
\prompt{In}{incolor}{18}{\boxspacing}
\begin{Verbatim}[commandchars=\\\{\}]
\PY{n+nb}{round}\PY{p}{(}\PY{n}{\PYZus{}}\PY{p}{,}\PY{l+m+mi}{8}\PY{p}{)}
\end{Verbatim}
\end{tcolorbox}
 
            
\prompt{Out}{outcolor}{18}{}
    
    $\displaystyle 19248.37563115 + i$

    

    Una expresi�n como la anterior puede asignarse a una variable, en este
caso la variable la llamaremos \(x\).

    \begin{tcolorbox}[breakable, size=fbox, boxrule=1pt, pad at break*=1mm,colback=cellbackground, colframe=cellborder]
\prompt{In}{incolor}{19}{\boxspacing}
\begin{Verbatim}[commandchars=\\\{\}]
\PY{n}{x}\PY{o}{=}\PY{n}{sqrt}\PY{p}{(}\PY{l+m+mi}{8}\PY{p}{)}\PY{o}{/}\PY{l+m+mi}{3}\PY{o}{+}\PY{n}{ln}\PY{p}{(}\PY{n}{E}\PY{o}{+}\PY{l+m+mi}{1}\PY{p}{)}\PY{o}{+}\PY{n}{log}\PY{p}{(}\PY{l+m+mi}{3}\PY{p}{,}\PY{l+m+mi}{10}\PY{p}{)}\PY{o}{+}\PY{n}{E}\PY{o}{*}\PY{o}{*}\PY{p}{(}\PY{n}{pi}\PY{o}{*}\PY{o}{*}\PY{l+m+mi}{2}\PY{p}{)}\PY{o}{+}\PY{n}{exp}\PY{p}{(}\PY{n}{pi}\PY{p}{)}\PY{o}{*}\PY{o}{*}\PY{l+m+mi}{2}
\PY{n}{x}
\end{Verbatim}
\end{tcolorbox}
 
            
\prompt{Out}{outcolor}{19}{}
    
    $\displaystyle \frac{\log{\left(3 \right)}}{\log{\left(10 \right)}} + \frac{2 \sqrt{2}}{3} + \log{\left(1 + e \right)} + e^{2 \pi} + e^{\pi^{2}}$

    

    \begin{tcolorbox}[breakable, size=fbox, boxrule=1pt, pad at break*=1mm,colback=cellbackground, colframe=cellborder]
\prompt{In}{incolor}{20}{\boxspacing}
\begin{Verbatim}[commandchars=\\\{\}]
\PY{n+nb}{float}\PY{p}{(}\PY{n}{\PYZus{}}\PY{p}{)}
\end{Verbatim}
\end{tcolorbox}

            \begin{tcolorbox}[breakable, size=fbox, boxrule=.5pt, pad at break*=1mm, opacityfill=0]
\prompt{Out}{outcolor}{20}{\boxspacing}
\begin{Verbatim}[commandchars=\\\{\}]
19871.91392187373
\end{Verbatim}
\end{tcolorbox}
        
    \begin{tcolorbox}[breakable, size=fbox, boxrule=1pt, pad at break*=1mm,colback=cellbackground, colframe=cellborder]
\prompt{In}{incolor}{21}{\boxspacing}
\begin{Verbatim}[commandchars=\\\{\}]
\PY{n+nb}{round}\PY{p}{(}\PY{n}{x}\PY{p}{,} \PY{l+m+mi}{2}\PY{p}{)}
\end{Verbatim}
\end{tcolorbox}
 
            
\prompt{Out}{outcolor}{21}{}
    
    $\displaystyle 19871.91$

    

    \begin{tcolorbox}[breakable, size=fbox, boxrule=1pt, pad at break*=1mm,colback=cellbackground, colframe=cellborder]
\prompt{In}{incolor}{22}{\boxspacing}
\begin{Verbatim}[commandchars=\\\{\}]
\PY{n}{N}\PY{p}{(}\PY{n}{\PYZus{}}\PY{p}{,}\PY{l+m+mi}{2}\PY{p}{)}
\end{Verbatim}
\end{tcolorbox}
 
            
\prompt{Out}{outcolor}{22}{}
    
    $\displaystyle 2.0 \cdot 10^{4}$

    

    SymPy tambi�n se entiende con los números imaginarios, por ejemplo:
\(\sqrt{-1}+2i\)

    \begin{tcolorbox}[breakable, size=fbox, boxrule=1pt, pad at break*=1mm,colback=cellbackground, colframe=cellborder]
\prompt{In}{incolor}{23}{\boxspacing}
\begin{Verbatim}[commandchars=\\\{\}]
\PY{n}{sqrt}\PY{p}{(}\PY{o}{\PYZhy{}}\PY{l+m+mi}{1}\PY{p}{)}\PY{o}{+}\PY{l+m+mi}{2}\PY{o}{*}\PY{n}{I}
\end{Verbatim}
\end{tcolorbox}
 
            
\prompt{Out}{outcolor}{23}{}
    
    $\displaystyle 3 i$

    

    Una de las ecuaciones m�s bonitas de las matem�ticas:

    \(e^{2\pi i} =1\)

    \begin{tcolorbox}[breakable, size=fbox, boxrule=1pt, pad at break*=1mm,colback=cellbackground, colframe=cellborder]
\prompt{In}{incolor}{24}{\boxspacing}
\begin{Verbatim}[commandchars=\\\{\}]
\PY{n}{exp}\PY{p}{(}\PY{l+m+mi}{2}\PY{o}{*}\PY{n}{pi}\PY{o}{*}\PY{n}{I}\PY{p}{)}\PY{o}{\PYZhy{}}\PY{l+m+mi}{1}
\end{Verbatim}
\end{tcolorbox}
 
            
\prompt{Out}{outcolor}{24}{}
    
    $\displaystyle 0$

    

    A diferencia de muchos sistemas de manipulaci�n simb�lica, en SymPy las
variables deben definirse antes de usarse.

    \begin{tcolorbox}[breakable, size=fbox, boxrule=1pt, pad at break*=1mm,colback=cellbackground, colframe=cellborder]
\prompt{In}{incolor}{25}{\boxspacing}
\begin{Verbatim}[commandchars=\\\{\}]
\PY{n}{x}\PY{p}{,} \PY{n}{y}  \PY{o}{=} \PY{n}{symbols}\PY{p}{(}\PY{l+s+s1}{\PYZsq{}}\PY{l+s+s1}{x y}\PY{l+s+s1}{\PYZsq{}}\PY{p}{)}
\end{Verbatim}
\end{tcolorbox}

    \begin{tcolorbox}[breakable, size=fbox, boxrule=1pt, pad at break*=1mm,colback=cellbackground, colframe=cellborder]
\prompt{In}{incolor}{26}{\boxspacing}
\begin{Verbatim}[commandchars=\\\{\}]
\PY{n}{r} \PY{o}{=} \PY{n}{x} \PY{o}{+} \PY{l+m+mi}{2}\PY{o}{*}\PY{n}{y}
\PY{n}{r}
\end{Verbatim}
\end{tcolorbox}
 
            
\prompt{Out}{outcolor}{26}{}
    
    $\displaystyle x + 2 y$

    

    \begin{tcolorbox}[breakable, size=fbox, boxrule=1pt, pad at break*=1mm,colback=cellbackground, colframe=cellborder]
\prompt{In}{incolor}{27}{\boxspacing}
\begin{Verbatim}[commandchars=\\\{\}]
\PY{n}{p}\PY{o}{=}\PY{n}{x}\PY{o}{*}\PY{o}{*}\PY{l+m+mi}{2}\PY{o}{*}\PY{n}{r}\PY{o}{*}\PY{o}{*}\PY{l+m+mi}{3}
\PY{n}{p}
\end{Verbatim}
\end{tcolorbox}
 
            
\prompt{Out}{outcolor}{27}{}
    
    $\displaystyle x^{2} \left(x + 2 y\right)^{3}$

    

    Ahora recurriremos a una de las facilidades de estructura de datos que
nos ofrece el sistema para agrupar objetos matem�ticos: las listas.

Las listas son un arreglo de cantidades que se escriben entre corchetes
y los objetos deben separarse con comas.

    \begin{tcolorbox}[breakable, size=fbox, boxrule=1pt, pad at break*=1mm,colback=cellbackground, colframe=cellborder]
\prompt{In}{incolor}{28}{\boxspacing}
\begin{Verbatim}[commandchars=\\\{\}]
\PY{p}{[}\PY{n}{sin}\PY{p}{(}\PY{n}{pi}\PY{o}{/}\PY{l+m+mi}{3}\PY{p}{)}\PY{p}{,}\PY{n}{cos}\PY{p}{(}\PY{n}{pi}\PY{o}{/}\PY{l+m+mi}{3}\PY{p}{)}\PY{p}{,}\PY{n}{tan}\PY{p}{(}\PY{n}{pi}\PY{o}{/}\PY{l+m+mi}{3}\PY{p}{)}\PY{p}{]}
\end{Verbatim}
\end{tcolorbox}

            \begin{tcolorbox}[breakable, size=fbox, boxrule=.5pt, pad at break*=1mm, opacityfill=0]
\prompt{Out}{outcolor}{28}{\boxspacing}
\begin{Verbatim}[commandchars=\\\{\}]
[sqrt(3)/2, 1/2, sqrt(3)]
\end{Verbatim}
\end{tcolorbox}
        
    Podemos asignal la lista a una variable

    \begin{tcolorbox}[breakable, size=fbox, boxrule=1pt, pad at break*=1mm,colback=cellbackground, colframe=cellborder]
\prompt{In}{incolor}{29}{\boxspacing}
\begin{Verbatim}[commandchars=\\\{\}]
\PY{n}{L} \PY{o}{=} \PY{p}{[}\PY{n}{sin}\PY{p}{(}\PY{n}{pi}\PY{o}{/}\PY{l+m+mi}{3}\PY{p}{)}\PY{p}{,}\PY{n}{cos}\PY{p}{(}\PY{n}{pi}\PY{o}{/}\PY{l+m+mi}{3}\PY{p}{)}\PY{p}{,}\PY{n}{tan}\PY{p}{(}\PY{n}{pi}\PY{o}{/}\PY{l+m+mi}{3}\PY{p}{)}\PY{p}{]}
\end{Verbatim}
\end{tcolorbox}

    Para luego aplicar la funci�n evalf() a cada elemento de la lista

    \begin{tcolorbox}[breakable, size=fbox, boxrule=1pt, pad at break*=1mm,colback=cellbackground, colframe=cellborder]
\prompt{In}{incolor}{30}{\boxspacing}
\begin{Verbatim}[commandchars=\\\{\}]
\PY{p}{[}\PY{n}{expr}\PY{o}{.}\PY{n}{evalf}\PY{p}{(}\PY{p}{)} \PY{k}{for} \PY{n}{expr} \PY{o+ow}{in} \PY{n}{L}\PY{p}{]}
\end{Verbatim}
\end{tcolorbox}

            \begin{tcolorbox}[breakable, size=fbox, boxrule=.5pt, pad at break*=1mm, opacityfill=0]
\prompt{Out}{outcolor}{30}{\boxspacing}
\begin{Verbatim}[commandchars=\\\{\}]
[0.866025403784439, 0.500000000000000, 1.73205080756888]
\end{Verbatim}
\end{tcolorbox}
        
    El primer elemento de la lista L es:

    \begin{tcolorbox}[breakable, size=fbox, boxrule=1pt, pad at break*=1mm,colback=cellbackground, colframe=cellborder]
\prompt{In}{incolor}{31}{\boxspacing}
\begin{Verbatim}[commandchars=\\\{\}]
\PY{n}{L}\PY{p}{[}\PY{l+m+mi}{0}\PY{p}{]}
\end{Verbatim}
\end{tcolorbox}
 
            
\prompt{Out}{outcolor}{31}{}
    
    $\displaystyle \frac{\sqrt{3}}{2}$

    

    De manera que se puede hacer operaciones con los elementos de las listas

    \begin{tcolorbox}[breakable, size=fbox, boxrule=1pt, pad at break*=1mm,colback=cellbackground, colframe=cellborder]
\prompt{In}{incolor}{32}{\boxspacing}
\begin{Verbatim}[commandchars=\\\{\}]
\PY{p}{(}\PY{n}{L}\PY{p}{[}\PY{l+m+mi}{0}\PY{p}{]}\PY{o}{+}\PY{n}{L}\PY{p}{[}\PY{l+m+mi}{1}\PY{p}{]}\PY{p}{)}\PY{o}{/}\PY{n}{L}\PY{p}{[}\PY{l+m+mi}{2}\PY{p}{]}
\end{Verbatim}
\end{tcolorbox}
 
            
\prompt{Out}{outcolor}{32}{}
    
    $\displaystyle \frac{\sqrt{3} \cdot \left(\frac{1}{2} + \frac{\sqrt{3}}{2}\right)}{3}$

    

    Cuando necesitemos generar una lista por medio de alguna regla
espec�fica o f�rmula podemos hacer lo siguiente

    \begin{tcolorbox}[breakable, size=fbox, boxrule=1pt, pad at break*=1mm,colback=cellbackground, colframe=cellborder]
\prompt{In}{incolor}{33}{\boxspacing}
\begin{Verbatim}[commandchars=\\\{\}]
\PY{c+c1}{\PYZsh{} Definir la variable y el rango}
\PY{n}{t} \PY{o}{=} \PY{n}{symbols}\PY{p}{(}\PY{l+s+s1}{\PYZsq{}}\PY{l+s+s1}{t}\PY{l+s+s1}{\PYZsq{}}\PY{p}{,}\PY{n}{real}\PY{o}{=}\PY{k+kc}{True}\PY{p}{)}
\PY{n}{ini} \PY{o}{=} \PY{l+m+mi}{1}
\PY{n}{fin} \PY{o}{=} \PY{l+m+mi}{6}

\PY{c+c1}{\PYZsh{} Crear la lista de expresiones con una comprensi�n de lista}
\PY{n}{lista} \PY{o}{=} \PY{p}{[}\PY{n}{exp}\PY{p}{(}\PY{n}{t}\PY{o}{*}\PY{n}{x}\PY{p}{)} \PY{k}{for} \PY{n}{x} \PY{o+ow}{in} \PY{n+nb}{range}\PY{p}{(}\PY{n}{ini}\PY{p}{,} \PY{n}{fin} \PY{o}{+} \PY{l+m+mi}{1}\PY{p}{)}\PY{p}{]}

\PY{n+nb}{print}\PY{p}{(}\PY{n}{lista}\PY{p}{)}  \PY{c+c1}{\PYZsh{} Imprimir la lista de expresiones}
\end{Verbatim}
\end{tcolorbox}

    \begin{Verbatim}[commandchars=\\\{\}]
[exp(t), exp(2*t), exp(3*t), exp(4*t), exp(5*t), exp(6*t)]
    \end{Verbatim}

    Es posible aplicar una funci�n a cada elemento de la lista, en este caso
a cada elemento le aplicaremos la funci�n \(\ln(x)\)

    \begin{tcolorbox}[breakable, size=fbox, boxrule=1pt, pad at break*=1mm,colback=cellbackground, colframe=cellborder]
\prompt{In}{incolor}{34}{\boxspacing}
\begin{Verbatim}[commandchars=\\\{\}]
\PY{n}{lista2}\PY{o}{=}\PY{p}{[}\PY{p}{(}\PY{n}{log}\PY{p}{(}\PY{n}{expr}\PY{p}{)}\PY{p}{)} \PY{k}{for} \PY{n}{expr} \PY{o+ow}{in} \PY{n}{lista}\PY{p}{]}
\PY{n}{lista2}
\end{Verbatim}
\end{tcolorbox}

            \begin{tcolorbox}[breakable, size=fbox, boxrule=.5pt, pad at break*=1mm, opacityfill=0]
\prompt{Out}{outcolor}{34}{\boxspacing}
\begin{Verbatim}[commandchars=\\\{\}]
[t, 2*t, 3*t, 4*t, 5*t, 6*t]
\end{Verbatim}
\end{tcolorbox}
        
    Si lo que se quiere es sumar los elementos:

    \begin{tcolorbox}[breakable, size=fbox, boxrule=1pt, pad at break*=1mm,colback=cellbackground, colframe=cellborder]
\prompt{In}{incolor}{35}{\boxspacing}
\begin{Verbatim}[commandchars=\\\{\}]
\PY{n}{suma} \PY{o}{=} \PY{n+nb}{sum}\PY{p}{(}\PY{n}{lista2}\PY{p}{)}
\PY{n}{suma}
\end{Verbatim}
\end{tcolorbox}
 
            
\prompt{Out}{outcolor}{35}{}
    
    $\displaystyle 21 t$

    

    Una utilidad que le podemos dar a las listas es que nos permite definir
vectores. Esto se puede hacer recurriendo al hecho de que los vectores
se pueden ver como los elementos de una matriz fila o columna. Trabajar
con matrices es sencillo

    \begin{tcolorbox}[breakable, size=fbox, boxrule=1pt, pad at break*=1mm,colback=cellbackground, colframe=cellborder]
\prompt{In}{incolor}{36}{\boxspacing}
\begin{Verbatim}[commandchars=\\\{\}]
\PY{n}{A} \PY{o}{=} \PY{n}{Matrix}\PY{p}{(}\PY{p}{[}\PY{p}{[}\PY{l+m+mi}{4}\PY{p}{,} \PY{l+m+mi}{5}\PY{p}{,} \PY{l+m+mi}{6}\PY{p}{]}\PY{p}{]}\PY{p}{)} \PY{c+c1}{\PYZsh{} un vector fila 1x3}
\PY{n}{A}
\end{Verbatim}
\end{tcolorbox}
 
            
\prompt{Out}{outcolor}{36}{}
    
    $\displaystyle \left[\begin{matrix}4 & 5 & 6\end{matrix}\right]$

    

    \begin{tcolorbox}[breakable, size=fbox, boxrule=1pt, pad at break*=1mm,colback=cellbackground, colframe=cellborder]
\prompt{In}{incolor}{37}{\boxspacing}
\begin{Verbatim}[commandchars=\\\{\}]
\PY{n}{B} \PY{o}{=} \PY{n}{Matrix} \PY{p}{(}\PY{p}{[}\PY{p}{[}\PY{l+m+mi}{7}\PY{p}{]} \PY{p}{,}\PY{p}{[}\PY{l+m+mi}{8}\PY{p}{]} \PY{p}{,} \PY{p}{[}\PY{l+m+mi}{9}\PY{p}{]}\PY{p}{]}\PY{p}{)} \PY{c+c1}{\PYZsh{} un vector columna 3x1}
\PY{n}{B}
\end{Verbatim}
\end{tcolorbox}
 
            
\prompt{Out}{outcolor}{37}{}
    
    $\displaystyle \left[\begin{matrix}7\\8\\9\end{matrix}\right]$

    

    \begin{tcolorbox}[breakable, size=fbox, boxrule=1pt, pad at break*=1mm,colback=cellbackground, colframe=cellborder]
\prompt{In}{incolor}{38}{\boxspacing}
\begin{Verbatim}[commandchars=\\\{\}]
\PY{n}{B}\PY{o}{.}\PY{n}{T} \PY{c+c1}{\PYZsh{} vector traspuesta de B}
\end{Verbatim}
\end{tcolorbox}
 
            
\prompt{Out}{outcolor}{38}{}
    
    $\displaystyle \left[\begin{matrix}7 & 8 & 9\end{matrix}\right]$

    

    \begin{tcolorbox}[breakable, size=fbox, boxrule=1pt, pad at break*=1mm,colback=cellbackground, colframe=cellborder]
\prompt{In}{incolor}{39}{\boxspacing}
\begin{Verbatim}[commandchars=\\\{\}]
\PY{n}{A}\PY{p}{[}\PY{l+m+mi}{0}\PY{p}{]} \PY{c+c1}{\PYZsh{} Primera componente del vector A (�ndice 0)}
\end{Verbatim}
\end{tcolorbox}
 
            
\prompt{Out}{outcolor}{39}{}
    
    $\displaystyle 4$

    

    \begin{tcolorbox}[breakable, size=fbox, boxrule=1pt, pad at break*=1mm,colback=cellbackground, colframe=cellborder]
\prompt{In}{incolor}{40}{\boxspacing}
\begin{Verbatim}[commandchars=\\\{\}]
\PY{n}{A}\PY{o}{.}\PY{n}{norm}\PY{p}{(}\PY{p}{)} \PY{c+c1}{\PYZsh{} norma del vector A}
\end{Verbatim}
\end{tcolorbox}
 
            
\prompt{Out}{outcolor}{40}{}
    
    $\displaystyle \sqrt{77}$

    

    \begin{tcolorbox}[breakable, size=fbox, boxrule=1pt, pad at break*=1mm,colback=cellbackground, colframe=cellborder]
\prompt{In}{incolor}{41}{\boxspacing}
\begin{Verbatim}[commandchars=\\\{\}]
\PY{n}{Ahat} \PY{o}{=} \PY{n}{A}\PY{o}{/}\PY{n}{A}\PY{o}{.}\PY{n}{norm}\PY{p}{(}\PY{p}{)} \PY{c+c1}{\PYZsh{} vector unitario asociada a A}
\PY{n}{Ahat}
\end{Verbatim}
\end{tcolorbox}
 
            
\prompt{Out}{outcolor}{41}{}
    
    $\displaystyle \left[\begin{matrix}\frac{4 \sqrt{77}}{77} & \frac{5 \sqrt{77}}{77} & \frac{6 \sqrt{77}}{77}\end{matrix}\right]$

    

    \begin{tcolorbox}[breakable, size=fbox, boxrule=1pt, pad at break*=1mm,colback=cellbackground, colframe=cellborder]
\prompt{In}{incolor}{42}{\boxspacing}
\begin{Verbatim}[commandchars=\\\{\}]
\PY{n}{Ahat}\PY{o}{.}\PY{n}{norm}\PY{p}{(}\PY{p}{)}
\end{Verbatim}
\end{tcolorbox}
 
            
\prompt{Out}{outcolor}{42}{}
    
    $\displaystyle 1$

    

    Definamos los siguientes vectores \[
\mathbf{a}=2 \mathbf{i}+4 \mathbf{j}+6 \mathbf{k}, \quad \mathbf{b}=5 \mathbf{i}+7 \mathbf{j}+9 \mathbf{k} \,\,\text { y }\,\,  \mathbf{c}=\mathbf{i}+3 \mathbf{j}
\]

    \begin{tcolorbox}[breakable, size=fbox, boxrule=1pt, pad at break*=1mm,colback=cellbackground, colframe=cellborder]
\prompt{In}{incolor}{43}{\boxspacing}
\begin{Verbatim}[commandchars=\\\{\}]
\PY{n}{a} \PY{o}{=} \PY{n}{Matrix}\PY{p}{(}\PY{p}{[}\PY{l+m+mi}{2}\PY{p}{,}\PY{l+m+mi}{4}\PY{p}{,}\PY{l+m+mi}{6}\PY{p}{]}\PY{p}{)}
\PY{n}{b} \PY{o}{=} \PY{n}{Matrix}\PY{p}{(}\PY{p}{[}\PY{l+m+mi}{5}\PY{p}{,}\PY{l+m+mi}{7}\PY{p}{,}\PY{l+m+mi}{9}\PY{p}{]}\PY{p}{)}
\PY{n}{c} \PY{o}{=} \PY{n}{Matrix}\PY{p}{(}\PY{p}{[}\PY{l+m+mi}{1}\PY{p}{,}\PY{l+m+mi}{3}\PY{p}{,}\PY{l+m+mi}{0}\PY{p}{]}\PY{p}{)}
\end{Verbatim}
\end{tcolorbox}

    \begin{tcolorbox}[breakable, size=fbox, boxrule=1pt, pad at break*=1mm,colback=cellbackground, colframe=cellborder]
\prompt{In}{incolor}{44}{\boxspacing}
\begin{Verbatim}[commandchars=\\\{\}]
\PY{n}{a}\PY{o}{+}\PY{n}{b}\PY{o}{+}\PY{n}{c}
\end{Verbatim}
\end{tcolorbox}
 
            
\prompt{Out}{outcolor}{44}{}
    
    $\displaystyle \left[\begin{matrix}8\\14\\15\end{matrix}\right]$

    

    \begin{tcolorbox}[breakable, size=fbox, boxrule=1pt, pad at break*=1mm,colback=cellbackground, colframe=cellborder]
\prompt{In}{incolor}{45}{\boxspacing}
\begin{Verbatim}[commandchars=\\\{\}]
\PY{l+m+mi}{3}\PY{o}{*}\PY{n}{a}\PY{o}{+}\PY{l+m+mi}{5}\PY{o}{*}\PY{n}{b}\PY{o}{\PYZhy{}}\PY{n}{c}
\end{Verbatim}
\end{tcolorbox}
 
            
\prompt{Out}{outcolor}{45}{}
    
    $\displaystyle \left[\begin{matrix}30\\44\\63\end{matrix}\right]$

    

    Recordemos que el primer elemento ser� la primer componente del vector

    El producto escalar: \[
\mathbf{a} \cdot \mathbf{b} \equiv a_x b_x+a_y b_y+a_z b_z \equiv\|\mathbf{a}\|\|\mathbf{b}\| \cos (\varphi) \in \mathbb{R}
\]

    \begin{tcolorbox}[breakable, size=fbox, boxrule=1pt, pad at break*=1mm,colback=cellbackground, colframe=cellborder]
\prompt{In}{incolor}{46}{\boxspacing}
\begin{Verbatim}[commandchars=\\\{\}]
\PY{n}{a}\PY{o}{.}\PY{n}{dot}\PY{p}{(}\PY{n}{b}\PY{p}{)}
\end{Verbatim}
\end{tcolorbox}
 
            
\prompt{Out}{outcolor}{46}{}
    
    $\displaystyle 92$

    

    \begin{tcolorbox}[breakable, size=fbox, boxrule=1pt, pad at break*=1mm,colback=cellbackground, colframe=cellborder]
\prompt{In}{incolor}{47}{\boxspacing}
\begin{Verbatim}[commandchars=\\\{\}]
\PY{n}{b}\PY{o}{.}\PY{n}{dot}\PY{p}{(}\PY{n}{a}\PY{p}{)}
\end{Verbatim}
\end{tcolorbox}
 
            
\prompt{Out}{outcolor}{47}{}
    
    $\displaystyle 92$

    

    El �ngulo entre los vectores \[
\varphi= \arccos\left[ \frac{\mathbf{a} \cdot \mathbf{b}}{\|\mathbf{a}\|\|\mathbf{b}\|}\right]
\]

    \begin{tcolorbox}[breakable, size=fbox, boxrule=1pt, pad at break*=1mm,colback=cellbackground, colframe=cellborder]
\prompt{In}{incolor}{48}{\boxspacing}
\begin{Verbatim}[commandchars=\\\{\}]
\PY{n}{acos}\PY{p}{(}\PY{n}{a}\PY{o}{.}\PY{n}{dot}\PY{p}{(}\PY{n}{b}\PY{p}{)}\PY{o}{/}\PY{p}{(}\PY{n}{a}\PY{o}{.}\PY{n}{norm}\PY{p}{(}\PY{p}{)}\PY{o}{*}\PY{n}{b}\PY{o}{.}\PY{n}{norm}\PY{p}{(}\PY{p}{)}\PY{p}{)}\PY{p}{)}\PY{o}{.}\PY{n}{round}\PY{p}{(}\PY{l+m+mi}{3}\PY{p}{)} \PY{c+c1}{\PYZsh{} En radianes}
\end{Verbatim}
\end{tcolorbox}
 
            
\prompt{Out}{outcolor}{48}{}
    
    $\displaystyle 0.158$

    

    En grados ser�a:

    \begin{tcolorbox}[breakable, size=fbox, boxrule=1pt, pad at break*=1mm,colback=cellbackground, colframe=cellborder]
\prompt{In}{incolor}{49}{\boxspacing}
\begin{Verbatim}[commandchars=\\\{\}]
\PY{p}{(}\PY{p}{(}\PY{n}{\PYZus{}}\PY{p}{)}\PY{o}{*}\PY{l+m+mi}{180}\PY{o}{/}\PY{n}{pi}\PY{p}{)}\PY{o}{.}\PY{n}{evalf}\PY{p}{(}\PY{l+m+mi}{4}\PY{p}{)} \PY{c+c1}{\PYZsh{} grados}
\end{Verbatim}
\end{tcolorbox}
 
            
\prompt{Out}{outcolor}{49}{}
    
    $\displaystyle 9.052$

    

    El producto vectorial de dos vectores en 3 dimensiones es: \[
\mathbf{a} \times \mathbf{b}=a_y b_z-a_z b_y,\,\, a_z b_x-a_x b_z, \,\, a_x b_y-a_y b_x
\]

    \begin{tcolorbox}[breakable, size=fbox, boxrule=1pt, pad at break*=1mm,colback=cellbackground, colframe=cellborder]
\prompt{In}{incolor}{50}{\boxspacing}
\begin{Verbatim}[commandchars=\\\{\}]
\PY{n}{a}\PY{o}{.}\PY{n}{cross}\PY{p}{(}\PY{n}{b}\PY{p}{)}
\end{Verbatim}
\end{tcolorbox}
 
            
\prompt{Out}{outcolor}{50}{}
    
    $\displaystyle \left[\begin{matrix}-6\\12\\-6\end{matrix}\right]$

    

    \begin{tcolorbox}[breakable, size=fbox, boxrule=1pt, pad at break*=1mm,colback=cellbackground, colframe=cellborder]
\prompt{In}{incolor}{51}{\boxspacing}
\begin{Verbatim}[commandchars=\\\{\}]
\PY{n}{b}\PY{o}{.}\PY{n}{cross}\PY{p}{(}\PY{n}{a}\PY{p}{)}
\end{Verbatim}
\end{tcolorbox}
 
            
\prompt{Out}{outcolor}{51}{}
    
    $\displaystyle \left[\begin{matrix}6\\-12\\6\end{matrix}\right]$

    

    El producto tripe: \[
\mathbf{c}\cdot \left(\mathbf{a} \times \mathbf{b}\right)
\]

    \begin{tcolorbox}[breakable, size=fbox, boxrule=1pt, pad at break*=1mm,colback=cellbackground, colframe=cellborder]
\prompt{In}{incolor}{52}{\boxspacing}
\begin{Verbatim}[commandchars=\\\{\}]
\PY{p}{(}\PY{n}{a}\PY{o}{.}\PY{n}{cross}\PY{p}{(}\PY{n}{b}\PY{p}{)}\PY{p}{)}\PY{o}{.}\PY{n}{dot}\PY{p}{(}\PY{n}{c}\PY{p}{)}
\end{Verbatim}
\end{tcolorbox}
 
            
\prompt{Out}{outcolor}{52}{}
    
    $\displaystyle 30$

    

    Las operaciones con vectores gen�ricos que se pueden hacer son:

    \begin{tcolorbox}[breakable, size=fbox, boxrule=1pt, pad at break*=1mm,colback=cellbackground, colframe=cellborder]
\prompt{In}{incolor}{53}{\boxspacing}
\begin{Verbatim}[commandchars=\\\{\}]
\PY{n}{\alpha}\PY{p}{,}\PY{n}{\beta}\PY{p}{,} \PY{n}{a1}\PY{p}{,} \PY{n}{a2}\PY{p}{,} \PY{n}{a3}\PY{p}{,} \PY{n}{b1}\PY{p}{,} \PY{n}{b2}\PY{p}{,} \PY{n}{b3}\PY{p}{,}\PY{n}{c1}\PY{p}{,} \PY{n}{c2}\PY{p}{,} \PY{n}{c3} \PY{o}{=} \PY{n}{symbols}\PY{p}{(}\PY{l+s+s1}{\PYZsq{}}\PY{l+s+s1}{\alpha \beta a1  a2  a3  b1  b2  b3 c1  c2  c3 }\PY{l+s+s1}{\PYZsq{}}\PY{p}{)} 
\end{Verbatim}
\end{tcolorbox}

    \begin{tcolorbox}[breakable, size=fbox, boxrule=1pt, pad at break*=1mm,colback=cellbackground, colframe=cellborder]
\prompt{In}{incolor}{54}{\boxspacing}
\begin{Verbatim}[commandchars=\\\{\}]
\PY{n}{a}\PY{o}{=} \PY{n}{Matrix}\PY{p}{(}\PY{p}{[}\PY{p}{[}\PY{n}{a1}\PY{p}{,}\PY{n}{a2}\PY{p}{,}\PY{n}{a3}\PY{p}{]}\PY{p}{]}\PY{p}{)}
\PY{n}{b}\PY{o}{=} \PY{n}{Matrix}\PY{p}{(}\PY{p}{[}\PY{p}{[}\PY{n}{b1}\PY{p}{,}\PY{n}{b2}\PY{p}{,}\PY{n}{b3}\PY{p}{]}\PY{p}{]}\PY{p}{)} 
\PY{n}{c}\PY{o}{=} \PY{n}{Matrix}\PY{p}{(}\PY{p}{[}\PY{p}{[}\PY{n}{c1}\PY{p}{,}\PY{n}{c2}\PY{p}{,}\PY{n}{c3}\PY{p}{]}\PY{p}{]}\PY{p}{)}
\PY{n}{O}\PY{o}{=} \PY{n}{Matrix}\PY{p}{(}\PY{p}{[}\PY{p}{[}\PY{l+m+mi}{0}\PY{p}{,}\PY{l+m+mi}{0}\PY{p}{,}\PY{l+m+mi}{0}\PY{p}{]}\PY{p}{]}\PY{p}{)}
\end{Verbatim}
\end{tcolorbox}

    \begin{tcolorbox}[breakable, size=fbox, boxrule=1pt, pad at break*=1mm,colback=cellbackground, colframe=cellborder]
\prompt{In}{incolor}{55}{\boxspacing}
\begin{Verbatim}[commandchars=\\\{\}]
\PY{n}{a}
\end{Verbatim}
\end{tcolorbox}
 
            
\prompt{Out}{outcolor}{55}{}
    
    $\displaystyle \left[\begin{matrix}a_{1} & a_{2} & a_{3}\end{matrix}\right]$

    

    \begin{tcolorbox}[breakable, size=fbox, boxrule=1pt, pad at break*=1mm,colback=cellbackground, colframe=cellborder]
\prompt{In}{incolor}{56}{\boxspacing}
\begin{Verbatim}[commandchars=\\\{\}]
\PY{n}{a}\PY{o}{+}\PY{n}{b}
\end{Verbatim}
\end{tcolorbox}
 
            
\prompt{Out}{outcolor}{56}{}
    
    $\displaystyle \left[\begin{matrix}a_{1} + b_{1} & a_{2} + b_{2} & a_{3} + b_{3}\end{matrix}\right]$

    

    \begin{tcolorbox}[breakable, size=fbox, boxrule=1pt, pad at break*=1mm,colback=cellbackground, colframe=cellborder]
\prompt{In}{incolor}{57}{\boxspacing}
\begin{Verbatim}[commandchars=\\\{\}]
\PY{p}{(}\PY{n}{a}\PY{o}{+}\PY{n}{b}\PY{p}{)}\PY{o}{+}\PY{n}{c}
\end{Verbatim}
\end{tcolorbox}
 
            
\prompt{Out}{outcolor}{57}{}
    
    $\displaystyle \left[\begin{matrix}a_{1} + b_{1} + c_{1} & a_{2} + b_{2} + c_{2} & a_{3} + b_{3} + c_{3}\end{matrix}\right]$

    

    \begin{tcolorbox}[breakable, size=fbox, boxrule=1pt, pad at break*=1mm,colback=cellbackground, colframe=cellborder]
\prompt{In}{incolor}{58}{\boxspacing}
\begin{Verbatim}[commandchars=\\\{\}]
\PY{n}{a}\PY{o}{+}\PY{n}{O}
\end{Verbatim}
\end{tcolorbox}
 
            
\prompt{Out}{outcolor}{58}{}
    
    $\displaystyle \left[\begin{matrix}a_{1} & a_{2} & a_{3}\end{matrix}\right]$

    

    \begin{tcolorbox}[breakable, size=fbox, boxrule=1pt, pad at break*=1mm,colback=cellbackground, colframe=cellborder]
\prompt{In}{incolor}{59}{\boxspacing}
\begin{Verbatim}[commandchars=\\\{\}]
\PY{n}{a}\PY{o}{\PYZhy{}}\PY{n}{a}
\end{Verbatim}
\end{tcolorbox}
 
            
\prompt{Out}{outcolor}{59}{}
    
    $\displaystyle \left[\begin{matrix}0 & 0 & 0\end{matrix}\right]$

    

    \begin{tcolorbox}[breakable, size=fbox, boxrule=1pt, pad at break*=1mm,colback=cellbackground, colframe=cellborder]
\prompt{In}{incolor}{60}{\boxspacing}
\begin{Verbatim}[commandchars=\\\{\}]
\PY{n}{\alpha}\PY{o}{*}\PY{p}{(}\PY{n}{a}\PY{o}{+}\PY{n}{b}\PY{p}{)}
\end{Verbatim}
\end{tcolorbox}
 
            
\prompt{Out}{outcolor}{60}{}
    
    $\displaystyle \left[\begin{matrix}\alpha \left(a_{1} + b_{1}\right) & \alpha \left(a_{2} + b_{2}\right) & \alpha \left(a_{3} + b_{3}\right)\end{matrix}\right]$

    

    \begin{tcolorbox}[breakable, size=fbox, boxrule=1pt, pad at break*=1mm,colback=cellbackground, colframe=cellborder]
\prompt{In}{incolor}{61}{\boxspacing}
\begin{Verbatim}[commandchars=\\\{\}]
\PY{p}{(}\PY{n}{\alpha}\PY{o}{+}\PY{n}{\beta}\PY{p}{)}\PY{o}{*}\PY{n}{a}
\end{Verbatim}
\end{tcolorbox}
 
            
\prompt{Out}{outcolor}{61}{}
    
    $\displaystyle \left[\begin{matrix}a_{1} \left(\alpha + \beta\right) & a_{2} \left(\alpha + \beta\right) & a_{3} \left(\alpha + \beta\right)\end{matrix}\right]$

    

    \begin{tcolorbox}[breakable, size=fbox, boxrule=1pt, pad at break*=1mm,colback=cellbackground, colframe=cellborder]
\prompt{In}{incolor}{62}{\boxspacing}
\begin{Verbatim}[commandchars=\\\{\}]
\PY{p}{(}\PY{n}{a}\PY{o}{.}\PY{n}{cross}\PY{p}{(}\PY{n}{b}\PY{p}{)}\PY{p}{)}\PY{o}{.}\PY{n}{dot}\PY{p}{(}\PY{n}{c}\PY{p}{)}
\end{Verbatim}
\end{tcolorbox}
 
            
\prompt{Out}{outcolor}{62}{}
    
    $\displaystyle c_{1} \left(a_{2} b_{3} - a_{3} b_{2}\right) + c_{2} \left(- a_{1} b_{3} + a_{3} b_{1}\right) + c_{3} \left(a_{1} b_{2} - a_{2} b_{1}\right)$

    

    Luego mostraremos una manera m�s eficiente de manejar vectores.




    
    
    
\end{document}
